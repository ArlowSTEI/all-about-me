% Options for packages loaded elsewhere
% Options for packages loaded elsewhere
\PassOptionsToPackage{unicode}{hyperref}
\PassOptionsToPackage{hyphens}{url}
\PassOptionsToPackage{dvipsnames,svgnames,x11names}{xcolor}
%
\documentclass[
  letterpaper,
  DIV=11,
  numbers=noendperiod]{scrreprt}
\usepackage{xcolor}
\usepackage{amsmath,amssymb}
\setcounter{secnumdepth}{5}
\usepackage{iftex}
\ifPDFTeX
  \usepackage[T1]{fontenc}
  \usepackage[utf8]{inputenc}
  \usepackage{textcomp} % provide euro and other symbols
\else % if luatex or xetex
  \usepackage{unicode-math} % this also loads fontspec
  \defaultfontfeatures{Scale=MatchLowercase}
  \defaultfontfeatures[\rmfamily]{Ligatures=TeX,Scale=1}
\fi
\usepackage{lmodern}
\ifPDFTeX\else
  % xetex/luatex font selection
\fi
% Use upquote if available, for straight quotes in verbatim environments
\IfFileExists{upquote.sty}{\usepackage{upquote}}{}
\IfFileExists{microtype.sty}{% use microtype if available
  \usepackage[]{microtype}
  \UseMicrotypeSet[protrusion]{basicmath} % disable protrusion for tt fonts
}{}
\makeatletter
\@ifundefined{KOMAClassName}{% if non-KOMA class
  \IfFileExists{parskip.sty}{%
    \usepackage{parskip}
  }{% else
    \setlength{\parindent}{0pt}
    \setlength{\parskip}{6pt plus 2pt minus 1pt}}
}{% if KOMA class
  \KOMAoptions{parskip=half}}
\makeatother
% Make \paragraph and \subparagraph free-standing
\makeatletter
\ifx\paragraph\undefined\else
  \let\oldparagraph\paragraph
  \renewcommand{\paragraph}{
    \@ifstar
      \xxxParagraphStar
      \xxxParagraphNoStar
  }
  \newcommand{\xxxParagraphStar}[1]{\oldparagraph*{#1}\mbox{}}
  \newcommand{\xxxParagraphNoStar}[1]{\oldparagraph{#1}\mbox{}}
\fi
\ifx\subparagraph\undefined\else
  \let\oldsubparagraph\subparagraph
  \renewcommand{\subparagraph}{
    \@ifstar
      \xxxSubParagraphStar
      \xxxSubParagraphNoStar
  }
  \newcommand{\xxxSubParagraphStar}[1]{\oldsubparagraph*{#1}\mbox{}}
  \newcommand{\xxxSubParagraphNoStar}[1]{\oldsubparagraph{#1}\mbox{}}
\fi
\makeatother


\usepackage{longtable,booktabs,array}
\usepackage{calc} % for calculating minipage widths
% Correct order of tables after \paragraph or \subparagraph
\usepackage{etoolbox}
\makeatletter
\patchcmd\longtable{\par}{\if@noskipsec\mbox{}\fi\par}{}{}
\makeatother
% Allow footnotes in longtable head/foot
\IfFileExists{footnotehyper.sty}{\usepackage{footnotehyper}}{\usepackage{footnote}}
\makesavenoteenv{longtable}
\usepackage{graphicx}
\makeatletter
\newsavebox\pandoc@box
\newcommand*\pandocbounded[1]{% scales image to fit in text height/width
  \sbox\pandoc@box{#1}%
  \Gscale@div\@tempa{\textheight}{\dimexpr\ht\pandoc@box+\dp\pandoc@box\relax}%
  \Gscale@div\@tempb{\linewidth}{\wd\pandoc@box}%
  \ifdim\@tempb\p@<\@tempa\p@\let\@tempa\@tempb\fi% select the smaller of both
  \ifdim\@tempa\p@<\p@\scalebox{\@tempa}{\usebox\pandoc@box}%
  \else\usebox{\pandoc@box}%
  \fi%
}
% Set default figure placement to htbp
\def\fps@figure{htbp}
\makeatother





\setlength{\emergencystretch}{3em} % prevent overfull lines

\providecommand{\tightlist}{%
  \setlength{\itemsep}{0pt}\setlength{\parskip}{0pt}}



 


\KOMAoption{captions}{tableheading}
\makeatletter
\@ifpackageloaded{bookmark}{}{\usepackage{bookmark}}
\makeatother
\makeatletter
\@ifpackageloaded{caption}{}{\usepackage{caption}}
\AtBeginDocument{%
\ifdefined\contentsname
  \renewcommand*\contentsname{Table of contents}
\else
  \newcommand\contentsname{Table of contents}
\fi
\ifdefined\listfigurename
  \renewcommand*\listfigurename{List of Figures}
\else
  \newcommand\listfigurename{List of Figures}
\fi
\ifdefined\listtablename
  \renewcommand*\listtablename{List of Tables}
\else
  \newcommand\listtablename{List of Tables}
\fi
\ifdefined\figurename
  \renewcommand*\figurename{Figure}
\else
  \newcommand\figurename{Figure}
\fi
\ifdefined\tablename
  \renewcommand*\tablename{Table}
\else
  \newcommand\tablename{Table}
\fi
}
\@ifpackageloaded{float}{}{\usepackage{float}}
\floatstyle{ruled}
\@ifundefined{c@chapter}{\newfloat{codelisting}{h}{lop}}{\newfloat{codelisting}{h}{lop}[chapter]}
\floatname{codelisting}{Listing}
\newcommand*\listoflistings{\listof{codelisting}{List of Listings}}
\makeatother
\makeatletter
\makeatother
\makeatletter
\@ifpackageloaded{caption}{}{\usepackage{caption}}
\@ifpackageloaded{subcaption}{}{\usepackage{subcaption}}
\makeatother
\usepackage{bookmark}
\IfFileExists{xurl.sty}{\usepackage{xurl}}{} % add URL line breaks if available
\urlstyle{same}
\hypersetup{
  pdftitle={Arlow Emmanuel Hergara},
  pdfauthor={13523161 Arlow Emmanuel Hergara},
  colorlinks=true,
  linkcolor={blue},
  filecolor={Maroon},
  citecolor={Blue},
  urlcolor={Blue},
  pdfcreator={LaTeX via pandoc}}


\title{Arlow Emmanuel Hergara}
\usepackage{etoolbox}
\makeatletter
\providecommand{\subtitle}[1]{% add subtitle to \maketitle
  \apptocmd{\@title}{\par {\large #1 \par}}{}{}
}
\makeatother
\subtitle{Portfolio Asesmen II-2100 KIPP}
\author{13523161 Arlow Emmanuel Hergara}
\date{2025-10-31}
\begin{document}
\maketitle

\renewcommand*\contentsname{Table of contents}
{
\hypersetup{linkcolor=}
\setcounter{tocdepth}{2}
\tableofcontents
}

\bookmarksetup{startatroot}

\chapter*{Selamat Berjumpa}\label{selamat-berjumpa}
\addcontentsline{toc}{chapter}{Selamat Berjumpa}

\markboth{Selamat Berjumpa}{Selamat Berjumpa}

\begin{figure}[H]

{\centering \includegraphics[width=9.5\linewidth,height=\textheight,keepaspectratio]{images/AZRL.png}

}

\caption{About Me}

\end{figure}%

Nama saya adalah Arlow Emmanuel Hergara. Orang biasa yang sedang
mengeksplorasi hidup dan semua intrikasi-intraksinya sembari
mengembangkan diri menjadi versi yang terbaik.

Saya memiliki ketertarikan dalam pemrograman terutama dalam game
development. Karena itu saya masuk ke dalam Sekolah Teknik Elektro dan
Informatika sebagai mahasiswa S1 Teknik Informatika. Sekarang, saya
sudah berada dalam semester 5.

\bookmarksetup{startatroot}

\chapter{UTS-1 All About Me}\label{uts-1-all-about-me}

\begin{figure}[H]

{\centering \includegraphics[width=9.5\linewidth,height=\textheight,keepaspectratio]{All_About_me/../images/AZRL.png}

}

\caption{About Me}

\end{figure}%

\section{Memahami Manusia}\label{memahami-manusia}

Tidak ada dari kita yang lahir mengetahui semuanya. Bahkan, kita masuk
ke dunia ini tanpa mengetahui apa-apa. Kita bertahan hidup hanya melalui
naluri dasar yang mendorong kita untuk melakukan aksi tertentu. Aksi
tersebut pun hanya dapat meminta bantuan dari manusia-manusia lain di
sekitar kita. Pada awalnya, semua manusia lahir tidak berdaya.

Namun, manusia adalah mahkluk yang terus belajar. Sejak awal kita lahir,
kita selalu mengobservasi dunia di sekitar kita. Kita mempelajari
bagaimana dunia itu bekerja. Kita juga coba berinteraksi dengan dunia
dan melihat bagaimana responnya. Dari yang kita amati kita dapatkan
pengetahuan dan dari yang kita lakukan kita dapatkan pengalaman.

Dalam mengeksplorasi dunia ini, kita menemukan bahwa dunia yang kita
alami berbeda antara satu orang dengan orang lain. Kita semua hidup
dalam dunia yang sama, tetapi dunia itu menunjukkan muka yang berbeda
pada masing-masing individu. Dengan demikian kita belajar hal-hal
berbeda, membangun pengetahun dan pengalaman masing-masing, dan menjadi
pribadi-pribadi yang unik. Tanpa menjalani keseluruhan hidup orang lain,
tidak mungkin kita memahami orang tersebut.

\section{Memahami Kesalahan}\label{memahami-kesalahan}

Dalam dunia ini, seringkali kita dengar kabar mengenai kesalahan yang
dilakukan oleh orang-orang. Terkadang, kita sendiri merasakan langsung
dampak dari kesalahan yang dilakukan oleh orang lain. Kesalahan ini
dapat menyakiti kita, bahkan hingga mendalam. Karena demikian, kita
dapat menanggap bahwa orang-orang tersebut merupakan orang yang buruk
dan memang berusaha untuk menyakiti kita.

Walaupun demikian, tidak ada orang di dunia ini yang secara aktif
berusaha menjadi pribadi yang buruk. Pribadi manusia tidak bisa
dilepaskan dari keadaan lingkungannya. Setiap orang berusaha melakukan
yang terbaik dengan semua informasi dan pengalaman yang dimilikinya.
Sayang saja, pilihan yang dibuat berdasarkan informasi dan pengalaman
yang diperoleh dalam suatu lingkungan belum tentu tepat dalam lingkungan
yang lain. Hal tersebut menggambarkan salah satu kejahatan utama dunia
ini. Manusia dipaksa membuat pilihan berdasarkan pengetahuan yang tidak
lengkap dan pilihan ini melekat menjadi pribadi barunya.

\section{Memahami Diri Sendiri}\label{memahami-diri-sendiri}

Pada dua subbab sebelumnya, saya sudah menggambarkan pandangan saya
terhadap dunia ini, khususnya pada sifat manusia. Namun, pandangan
tersebut hanya bagian kecil dari apa yang membuat diri saya. Agar dapat
mengenal saya dengan lebih lengkap, kita harus membahas detail-detail
yang lebih di permukaan dan mungkin kurang menarik mengenai saya. Untuk
menjelaskan hal tersebut, biarkan saya menggunakan nada yang lebih
kasual.

Halo, aku Arlow Emmanuel Hergara, orang yang biasa-biasa saja. Aku
adalah mahasiswa S1 Teknik Informatika di STEI ITB. Hobi-hobiku adalah
main game, makan, dan tidur. Selain dari hobi, aku juga memiliki minat
di bidang game development, software development, dan low level
programming. Aku juga memiliki kepribadian yang cukup unik (menurut
saya) dibandingkan dengan orang lain.

Aku adalah angkatan 2023 dari jurusan S1 Teknik Informatika di STEI ITB.
Saya masuk karena ketertarikan saya dengan programming. Dari sejak kecil
saya dipaparkan pada programming oleh ayahku dan hingga sekarang aku
masih tertarik pada hal-hal yang bisa saya buat. Namun jika aku perlu
jujur, aku merasa ketertarikan tersebut sudah tidak sekuat dulu semenjak
masuk ITB. Sepertinya tugas tidak berakhir yang aku kerjakan setiap hari
membuatku agak malas. Namun, aku tetap ingin semangat belajar mengenai
programming apapun \emph{challenge}-nya.

Hobi-hobiku (main game, makan, dan tidur) memang merupakan hobi yang
dimiliki banyak orang lain. Ketiga hal itu sering saya lakukan untuk
menghilangkan stres dari kesulitan sehari-hari (terkadang dilakuin
terlalu banyak sih). Jika harus spesifik, game yang aku banyak main
adalah game gacha, khususnya Genshin Impact dan game rhythm khususnya
osu! (ayo main bareng kalau ada yang main juga). Untuk makan, aku bisa
makan apa aja dan untuk tidur ya tidur.

Minat pertama yang aku dapatkan adalah game development. Bahkan game
development adalah hal yang membuat aku tertarik dengan programming. Aku
menjadi tertarik juga dalam software development setelah berusaha untuk
membuat program-program berhubungan dengan game development yang bukan
game. Minat aku terhadap low level programming muncul terakhir ketika
mencoba melakukan hal yang membutuhkannya. Dalam semua hal yang aku
minati, aku tertarik dengan bagaimana aku dapat membuat suatu sistem
bekerja seperti yang aku ingini.

Aku mengatakan bahwa kepribadianku unik tetapi aku tidak tahu seberapa
benar pernyataan itu. aku cenderung introvert yang tidak dapat
berinteraksi dengan siapa saja. Namun untuk orang yang aku kenal, aku
lumayan sering bercanda dan ngobrol. Aku juga seringkali melakukan
hal-hal yang tidak jelas yang orang lain dapat saja menganggap
mengesalkan. Sebelumnya aku sering takut melakukan hal-hal seperti itu
tapi seiringnya waktu aku belajar untuk tidak memikirkannya karena orang
pada umumnya tidak memerhatikan begitu banyak. Aku dulu orang yang suka
sedih, tetapi sekarang aku lagi berusaha menjadi seseorang yang lebih
ceria.

\bookmarksetup{startatroot}

\chapter{UTS-2 My Songs for You}\label{uts-2-my-songs-for-you}

\section{Musik yang Sedang Aku
Dengarkan}\label{musik-yang-sedang-aku-dengarkan}

\url{https://youtu.be/wWN7CaIQ2FU?si=lUU3BOX8Q6DYtetG}

\url{https://youtu.be/IZwVq37cOBk?si=cwOuogIIZ-2hfp7e}

\url{https://youtu.be/xPHjhcTK7gg?si=N-5LYuTnGL1Ehedq}

\section{Musik yang Aku Dengarkan Sejak
Lama}\label{musik-yang-aku-dengarkan-sejak-lama}

\url{https://youtu.be/pV1GoNi5mFs?si=CM5eZf3NFwQC1YW7}

\url{https://youtu.be/JprsKeAStcw?si=gu91EvJkcwhrBFEp}

\url{https://youtu.be/mpsDywA3wNI?si=ACt0uhPof_Js0E5I}

\bookmarksetup{startatroot}

\chapter{UTS-3 My Stories for You}\label{uts-3-my-stories-for-you}

\url{https://youtu.be/cn2hSXelQ0M?si=-xoBlNBDeCbjWpUF} Kisah ini
kebetulan saya temukan dalam bentuk MV sebuah musik. Namun saya tetap
menyukai kisah pengorbanan yang diceritakan dalamnya.

\bookmarksetup{startatroot}

\chapter{UTS-4 My SHAPE (Spiritual Gifts, Heart, Abilities, Personality,
Experiences)}\label{uts-4-my-shape-spiritual-gifts-heart-abilities-personality-experiences}

\section{SHAPE}\label{shape}

\subsection{S --- Spiritual Gifts (Karunia
Rohani)}\label{s-spiritual-gifts-karunia-rohani}

\textbf{Karunia utama}: Wisdom (hikmat) dan Teaching (mengajar) ---
kemampuan memahami struktur atau prinsip di balik suatu sistem, lalu
menerangkannya dengan logis agar orang lain mengerti dan bisa
menerapkannya.

\textbf{Kecenderungan alami}: suka merancang sistem atau konsep baru
(terutama lewat pemrograman dan desain logika), menemukan keteraturan
dari sesuatu yang rumit, serta menyalurkan wawasan itu lewat penjelasan
atau alat bantu (program, tulisan, atau diagram).

\textbf{Tanda-tanda karunia terlihat}:

\begin{itemize}
\tightlist
\item
  Orang lain sering meminta bantuanku untuk menjelaskan hal teknis atau
  memecahkan masalah logika.
\item
  Aku menikmati proses ``merancang cara berpikir'' --- bukan hanya hasil
  akhirnya.
\item
  Saat ideku berhasil membuat orang lain paham, aku merasa puas dan
  berarti.
\end{itemize}

\subsection{H --- Heart (Minat, Nilai, dan
Panggilan)}\label{h-heart-minat-nilai-dan-panggilan}

\textbf{Bidang yang dicintai}: teknologi, game, dan sistem berpikir yang
membuka kemungkinan baru. Nilai utama: eksplorasi, kebebasan berpikir,
dan membantu orang menemukan arah hidupnya.

\textbf{Panggilan pribadi}: menggunakan teknologi dan media interaktif
(seperti game) untuk memperluas cara orang belajar, berefleksi, dan
menemukan makna hidup.

\textbf{Masalah yang ingin diperbaiki}: banyak orang tersesat dalam
rutinitas tanpa arah, mereka butuh media yang menghidupkan kembali rasa
ingin tahu dan tujuan.

\textbf{Makna dalam karya}: ketika sebuah ide atau sistem yang saya buat
bisa memperluas kemungkinan, menyalakan rasa ingin tahu, dan memberi
jalan keluar bagi orang yang kehilangan arah.

\subsection{A --- Abilities (Kemampuan
Andal)}\label{a-abilities-kemampuan-andal}

\textbf{Kemampuan utama}:

\begin{itemize}
\tightlist
\item
  Pemrograman dalam berbagai bahasa dan paradigma (imperatif,
  fungsional, logika).
\item
  Desain sistem dan arsitektur perangkat lunak.
\item
  Debugging dan analisis kesalahan kompleks secara sistematis.
\item
  Koordinasi proyek dan pembagian tugas berbasis kemampuan tim.
\end{itemize}

\textbf{Gaya berpikir}: analitis dan terstruktur, mampu melihat
keterhubungan antarbagian dan mengoptimalkan sistem agar efisien dan
mudah dikembangkan.

\textbf{Kekuatan khas}: menyatukan konsep dari berbagai disiplin
(teknologi, logika, desain) menjadi satu sistem kerja yang elegan.

\textbf{Peran alami dalam tim}: desainer sistem dan penjaga kualitas,
orang yang memastikan proyek tetap konsisten dengan visi teknis dan
logika internalnya.

\subsection{P --- Personality (Kepribadian dan Gaya
Kerja)}\label{p-personality-kepribadian-dan-gaya-kerja}

\textbf{Gaya kerja}: fleksibel antara kerja mandiri dan kolaboratif,
mampu fokus mendalam saat dibutuhkan, tetapi juga siap berdiskusi untuk
mencari solusi yang paling rasional.

\textbf{Pola pengambilan keputusan}: berbasis logika dan prinsip, bukan
emosi sesaat. Sering memprioritaskan solusi ideal jangka panjang walau
memerlukan lebih banyak waktu dan energi.

\textbf{Kecenderungan khas}: perfeksionis terhadap sistem dan ide, ingin
setiap komponen memiliki alasan dan keterhubungan yang jelas.

\textbf{Sikap terhadap konflik}: tegas dan langsung menyampaikan
pendapat, terutama bila ada hal yang melanggar prinsip logika atau
efisiensi.

\textbf{Peran alami}: penalar strategis dan penjaga arah, memastikan
keputusan tidak sekadar cepat, tapi juga benar dalam jangka panjang.

\subsection{Experience (Pengalaman
Hidup)}\label{experience-pengalaman-hidup}

\textbf{Pengalaman pembentuk utama}: tantangan besar saat mencoba
membangun sistem operasi sendiri --- sebuah proyek yang menuntut
pemahaman mendalam tentang cara kerja komputer dari level paling dasar.

\textbf{Pelajaran dari kegagalan}: meski proyek itu belum berhasil,
pengalaman tersebut membentuk pola pikir sistematis dan menghargai
pentingnya fondasi yang kokoh dalam setiap rancangan. Kegagalan itu
menjadi pengingat bahwa visi besar membutuhkan kesabaran, disiplin, dan
kesediaan untuk memahami detail kecil sebelum membangun sesuatu yang
kompleks.

\textbf{Makna pribadi}: dari proses itu lahir tekad untuk terus memahami
bagaimana teknologi bekerja dari dalam, bukan sekadar menggunakannya.
Walau belum menemukan momen ``inilah panggilan saya,'' perjalanan
eksplorasi itu sendiri menjadi cara Anda mencari makna.

\section{Piagam Diri --- Arlow Emmanuel
Hergara}\label{piagam-diri-arlow-emmanuel-hergara}

\subsection{Misi Hidup (Life Mission)}\label{misi-hidup-life-mission}

Menjadi seseorang yang memajukan teknologi agar manusia dapat melakukan
hal-hal yang sebelumnya mustahil dilakukan --- menciptakan sistem yang
memperluas potensi, bukan membatasi.

\subsection{Nilai Inti (Core Values)}\label{nilai-inti-core-values}

\begin{enumerate}
\def\labelenumi{\arabic{enumi}.}
\tightlist
\item
  Rasionalitas dan kejujuran intelektual.
\item
  Empati --- kemajuan tidak boleh mengorbankan kesejahteraan orang lain.
\item
  Keberlanjutan --- setiap sistem harus dibangun agar tetap bermanfaat
  dan tidak merusak keseimbangan.
\end{enumerate}

\subsection{Prinsip Keputusan (Decision
Compass)}\label{prinsip-keputusan-decision-compass}

Saya akan menilai setiap tindakan dengan dua pertanyaan:

\begin{itemize}
\tightlist
\item
  Apakah ini masuk akal dan benar secara sistemik?
\item
  Apakah ini adil dan tidak memaksa orang untuk memberi lebih dari yang
  mereka miliki?
\end{itemize}

Jika suatu keputusan tidak lulus kedua pertanyaan ini, maka arah itu
bukan untuk saya.

\subsection{Peran Utama (Primary Role)}\label{peran-utama-primary-role}

Arsitek sistem dan penalar strategis --- seseorang yang merancang
fondasi logika dan teknologi agar orang lain dapat membangun di atasnya
dengan aman dan efisien.

\subsection{Gaya Pelayanan (Service
Style)}\label{gaya-pelayanan-service-style}

Mengajar dan membimbing melalui logika, menjelaskan hal sulit dengan
cara yang dapat dipahami, serta menciptakan alat yang membantu orang
lain memahami dunia mereka.

\subsection{Janji Hidup (Life Promise)}\label{janji-hidup-life-promise}

Saya berjanji untuk terus belajar, memahami dasar setiap hal, dan
menggunakan pemahaman itu untuk menciptakan teknologi yang memperluas
kebebasan manusia, bukan mempersempitnya.

\subsection{Batas Etis (Ethical
Boundaries)}\label{batas-etis-ethical-boundaries}

Saya tidak akan membuat sistem yang memaksa, mengeksploitasi, atau
menekan manusia demi efisiensi atau keuntungan. Teknologi harus melayani
manusia --- bukan sebaliknya.

\subsection{Narasi Diri (90 Detik)}\label{narasi-diri-90-detik}

Saya adalah seseorang yang mencintai logika dan sistem, dengan hasrat
untuk memahami bagaimana teknologi bekerja dan bagaimana ia dapat
memperluas kemampuan manusia. Saya menikmati proses merancang,
memprogram, dan membangun struktur yang tidak hanya efisien tetapi juga
adil bagi penggunanya. Saya percaya bahwa kemajuan sejati tidak boleh
memaksa siapa pun mengorbankan lebih dari yang mereka miliki. Pengalaman
saya mencoba membangun sistem operasi sendiri, meskipun belum berhasil,
mengajarkan pentingnya fondasi dan kesabaran dalam mewujudkan visi
besar. Melalui perjalanan ini, saya ingin terus berkontribusi dalam
pengembangan teknologi yang membantu manusia melakukan hal-hal yang
sebelumnya tidak mungkin, sambil tetap menjaga nilai-nilai kemanusiaan
di dalamnya.

\bookmarksetup{startatroot}

\chapter{UTS-5 My Personal Reviews}\label{uts-5-my-personal-reviews}

\section{Identifikasi}\label{identifikasi}

\begin{enumerate}
\def\labelenumi{\arabic{enumi}.}
\tightlist
\item
  \textbf{Nama Mahasiswa TUGAS:} Arlow Emmanuel Hergara (berdasarkan
  konten UTS-1 dan UTS-4)
\item
  \textbf{Nama Penilai:} Self-Assessment (Gemini)
\end{enumerate}

\begin{center}\rule{0.5\linewidth}{0.5pt}\end{center}

\section{Tinjauan Umum}\label{tinjauan-umum}

Portofolio UTS ini menunjukkan kualitas yang sangat bervariasi. Terdapat
kekuatan luar biasa dalam refleksi diri dan pemikiran terstruktur
(ditunjukkan pada UTS-1 dan UTS-4), yang menunjukkan kedalaman pemahaman
dan orisinalitas.

Namun, TUGAS UTS-2, UTS-3, dan UTS-5 tampaknya salah memahami inti dari
tugas tersebut. Alih-alih membuat atau menyampaikan ``pesan personal'',
tugas-tugas tersebut sebagian besar hanya berisi tautan ke karya orang
lain (UTS-2 \& UTS-3) atau mendelegasikan tugas analisis (UTS-5) ke AI
tanpa ada kontribusi analitis pribadi.

\begin{center}\rule{0.5\linewidth}{0.5pt}\end{center}

\section{Tinjauan Spesifik \& Skor
Persentase}\label{tinjauan-spesifik-skor-persentase}

Berikut adalah penilaian rinci untuk setiap TUGAS:

\subsection{UTS-1: All About Me}\label{uts-1-all-about-me-1}

\begin{itemize}
\tightlist
\item
  \textbf{Tinjauan:} Konten ini luar biasa. Dimulai dengan perenungan
  filosofis tentang ``Memahami Manusia'' dan ``Memahami Kesalahan''
  sebelum beralih ke perkenalan diri yang lebih kasual. Pendekatan ini
  sangat orisinal dan memberikan wawasan mendalam.
\item
  \textbf{Penilaian (Rubrik UTS-1):}

  \begin{itemize}
  \tightlist
  \item
    Orisinalitas (5/5): Sudut pandang sangat unik.
  \item
    Keterlibatan (4/5): Umumnya menarik.
  \item
    Humor (2/5): Kurang. Nada tulisan cenderung serius.
  \item
    Wawasan (Insight) (5/5): Memberi pemahaman mendalam.
  \end{itemize}
\item
  \textbf{Skor Persentase: 16 / 20 (80\%)}
\end{itemize}

\begin{center}\rule{0.5\linewidth}{0.5pt}\end{center}

\subsection{UTS-2: My Song for You}\label{uts-2-my-song-for-you}

\begin{itemize}
\tightlist
\item
  \textbf{Tinjauan:} Tugas ini meminta ``pesan berbentuk puisi, lago,
  dan/atau viodeo clip''{[}cite: 17{]}. Konten yang dikumpulkan hanya
  berupa enam tautan video YouTube tanpa konteks atau pesan pribadi.
\item
  \textbf{Penilaian (Rubrik UTS-2):}

  \begin{itemize}
  \tightlist
  \item
    Orisinalitas (1/5): Klise. Hanya daftar tautan.
  \item
    Keterlibatan (1/5): Tidak memikat.
  \item
    Humor (1/5): Tidak ada.
  \item
    Inspirasi (1/5): Tidak menginspirasi.
  \end{itemize}
\item
  \textbf{Skor Persentase: 4 / 20 (20\%)}
\end{itemize}

\begin{center}\rule{0.5\linewidth}{0.5pt}\end{center}

\subsection{UTS-3: My Stories for You}\label{uts-3-my-stories-for-you-1}

\begin{itemize}
\tightlist
\item
  \textbf{Tinjauan:} Tugas ini meminta ``kisah inspiratif dan menarik
  yang Anda ingin bagikan''{[}cite: 18{]}. Konten yang dikumpulkan
  adalah satu video musik dengan komentar singkat. Penulis tidak
  membagikan \emph{kisah mereka sendiri}.
\item
  \textbf{Penilaian (Rubrik UTS-3):}

  \begin{itemize}
  \tightlist
  \item
    Orisinalitas (1/5): Tidak ada pengembangan baru.
  \item
    Keterlibatan (1/5): Tidak menarik (dari sisi penulis).
  \item
    Pengembangan Narasi (1/5): Terputus dari cerita (personal).
  \item
    Inspirasi (2/5): Berusaha menginspirasi, namun dangkal.
  \end{itemize}
\item
  \textbf{Skor Persentase: 5 / 20 (25\%)}
\end{itemize}

\begin{center}\rule{0.5\linewidth}{0.5pt}\end{center}

\subsection{UTS-4: My SHAPE}\label{uts-4-my-shape}

\begin{itemize}
\tightlist
\item
  \textbf{Tinjauan:} Tugas ini meminta ``laporan siapa Anda berdasar
  hasil sebuah lembar kerja''{[}cite: 20{]}. Konten yang disajikan
  sangat baik, rinci, dan terstruktur. Penulis mengembangkannya menjadi
  ``Piagam Diri'' dan ``Narasi Diri''.
\item
  \textbf{Penilaian (Rubrik UTS-4):} \emph{(Rubrik di PDF tampaknya
  salah salin, penilaian ini menginterpretasikan kriteria dalam konteks
  laporan SHAPE).}

  \begin{itemize}
  \tightlist
  \item
    Orisinalitas (5/5): Pengembangan laporan sangat unik.
  \item
    Keterlibatan (5/5): Sangat memikat dan terstruktur.
  \item
    Pengembangan Narasi (5/5): Logis, dari analisis SHAPE ke Piagam
    Diri.
  \item
    Inspirasi (5/5): Sangat menginspirasi (mis. Misi Hidup, Batas Etis).
  \end{itemize}
\item
  \textbf{Skor Persentase: 20 / 20 (100\%)}
\end{itemize}

\begin{center}\rule{0.5\linewidth}{0.5pt}\end{center}

\subsection{UTS-5: My Personal
Reviews}\label{uts-5-my-personal-reviews-1}

\begin{itemize}
\tightlist
\item
  \textbf{Tinjauan:} Tugas ini adalah ``telaahan pesan personal
  berdasarkan rubrik''{[}cite: 21{]}. Konten yang dikumpulkan
  \emph{bukanlah} telaahan yang ditulis oleh penulis. Penulis
  mendokumentasikan proses menggunakan ChatGPT untuk menilai portofolio
  \emph{orang lain}.
\item
  \textbf{Penilaian (Rubrik UTS-5):}

  \begin{itemize}
  \tightlist
  \item
    Pemahaman Konsep (1/5): Tidak paham. Tugas didelegasikan.
  \item
    Analisis Kritis (1/5): Tidak kritis. Tidak ada analisis pribadi.
  \item
    Argumentasi (Logos) (1/5): Tidak logis.
  \item
    Etos \& Empati (1/5): Tidak tampak.
  \item
    Rekomendasi Perbaikan (1/5): Tidak ada (dari penulis).
  \end{itemize}
\item
  \textbf{Skor Persentase: 5 / 25 (20\%)}
\end{itemize}

\begin{center}\rule{0.5\linewidth}{0.5pt}\end{center}

\section{SKOR (Perhitungan Kontribusi
CPMK)}\label{skor-perhitungan-kontribusi-cpmk}

Berdasarkan \texttt{Tabel\ 1}, bobot untuk setiap tugas UTS adalah
sebagai berikut: * \textbf{UTS-1:} Bobot 6 (untuk CPMK-2) *
\textbf{UTS-2:} Bobot 7 (untuk CPMK-2) * \textbf{UTS-3:} Bobot 7 (untuk
CPMK-2) * \textbf{UTS-4:} Bobot 6 (untuk CPMK-2) * \textbf{UTS-5:} Bobot
10 (untuk CPMK-1)

Perhitungan kontribusi:
\texttt{(Skor\ Persentase\ /\ 100)\ *\ Bobot\ Tugas}

\begin{itemize}
\tightlist
\item
  \textbf{UTS-1 (CPMK-2):} (80\% / 100) * 6 = \textbf{4.8}
\item
  \textbf{UTS-2 (CPMK-2):} (20\% / 100) * 7 = \textbf{1.4}
\item
  \textbf{UTS-3 (CPMK-2):} (25\% / 100) * 7 = \textbf{1.75}
\item
  \textbf{UTS-4 (CPMK-2):} (100\% / 100) * 6 = \textbf{6.0}
\item
  \textbf{UTS-5 (CPMK-1):} (20\% / 100) * 10 = \textbf{2.0}
\end{itemize}

Tabel ini diisi sesuai format laporan{[}cite: 65{]}:

\begin{longtable}[]{@{}lcccc@{}}
\toprule\noalign{}
UTS & CPMK-1 & CPMK-2 & CPMK-3 & CPMK-4 \\
\midrule\noalign{}
\endhead
\bottomrule\noalign{}
\endlastfoot
UTS-1 & - & 4.8 & - & - \\
UTS-2 & - & 1.4 & - & - \\
UTS-3 & - & 1.75 & - & - \\
UTS-4 & - & 6.0 & - & - \\
UTS-5 & 2.0 & - & - & - \\
\textbf{Total Kontribusi} & \textbf{2.0} & \textbf{13.95} & \textbf{0} &
\textbf{0} \\
\emph{(Total Bobot UTS)} & \emph{(10)} & \emph{(26)} & \emph{(0)} &
\emph{(0)} \\
\end{longtable}

\bookmarksetup{startatroot}

\chapter{UAS-1 My Concepts}\label{uas-1-my-concepts}

\section{Computational Thinking}\label{computational-thinking}

Konsep yang ingin saya bawa adalah computational thinking.

Computational thinking adalah suatu pola pikir yang mampu membantu
seseorang menyelesaikan suatu masalah dengan cara yang jelas,
terstruktur, dan konsisten. Sebagai programmer, pola pikir ini menjadi
hal yang penting karena membantu menghasilkan solusi yang mampu
diimplementasikan ke dalam bentuk kode yang dapat dijalankan oleh suatu
komputer.

Pola pikir computational thinking terdiri dari 4 fondasi:

\begin{enumerate}
\def\labelenumi{\arabic{enumi}.}
\item
  \textbf{Decomposition:} Proses memecah suatu masalah menjadi
  masalah-masalah yang lebih kecil dan lebih mudah untuk diselesaikan.
\item
  \textbf{Abstraction:} Proses menghilangkangkan detail-detail tidak
  penting dalam suatu masalah agar dapat fokus pada inti permasalahan.
\item
  \textbf{Pattern Recognition:} Proses menemukan suatu pola dalam
  permasalahan yang ingin diselesaikan.
\item
  \textbf{Algorithmic Thinking:} Proses menentukan suatu algoritma atau
  langkah-langkah yang diperlukan untuk menyelesaikan masalah.
\end{enumerate}

\bookmarksetup{startatroot}

\chapter{UAS-3 My Opinions}\label{uas-3-my-opinions}

Opini yang saya berikan pada bagian ini adalah opini mengenai penggunaan
AI dalam pembuatan produk software (terutama dalam kode sumber).

Menurut saya, penggunaan AI perlu dilakukan bersama dengan dekomposisi
program menjadi komponen-komponen yang jelas serta pendefinisian kontrak
antar komponen yang sudah benar. Terdapat beberapa keuntungan ketika
menggunakan pendekatan ini ketika menggunakan AI untuk menulis kode.

\begin{enumerate}
\def\labelenumi{\arabic{enumi}.}
\item
  \textbf{Struktur kode yang lebih jelas:} Penentuan komponen-komponen
  perangkat lunak sebelum penulisan kode implementasi adalah hal yang
  umum dilakukan dalam pembuatan perangkat lunak. Hal ini dikarenakan
  penentuan struktur di awal membantu tim untuk membagi kerja secara
  jelas dan tetap menghasilkan komponen-komponen yang mampu berinteraksi
  dengan tepat. Untuk sekarang, AI masih belum mampu untuk menghasilkan
  struktur kode yang terbaik untuk suatu perangkat lunak dan lebih baik
  agar hal tersebut ditentukan sendiri oleh manusia.
\item
  \textbf{Mengurangi kesalahan:} Pendefinisian kontrak-kontrak yang
  jelas antar komponen memungkinkan pembuatan setiap komponen secara
  terpisah. Hal ini membantu AI sebab mengurangi hal-hal yang perlu
  dipikirkan oleh AI ketika menghasilkan suatu kode. Hal ini sama juga
  untuk developer manusia, tetapi orang cenderung lupa akan hal ini.
\item
  \textbf{Penggunaan AI lebih efisien:} Pembagian suatu program menjadi
  komponen-komponen memecahkan permasalahan keseluruhan program menjadi
  domain-domain yang lebih kecil. Hal ini dilakukan untuk membantu
  developer untuk fokus memecahkan masalah yang lebih kecil satu demi
  satu hingga menghasilkan program keseluruhan yang dapat memecahkan
  masalah yang lebih besar. AI juga mengambil keuntungan dari domain
  masalah yang lebih kecil sama seperti developer manusia. Ketika domain
  masalah lebih kecil, informasi yang harus disimpan oleh AI menjadi
  lebih sedikit. Hal ini berarti bahwa jumlah token yang digunakan oleh
  AI akan berkurang sehingga penggunaannya menjadi lebih efisien.
\end{enumerate}

\bookmarksetup{startatroot}

\chapter{UAS-3 My Innovations}\label{uas-3-my-innovations}

Inovasi yang saya kembangkan adalah

\bookmarksetup{startatroot}

\chapter{UAS-4 My Knowledge}\label{uas-4-my-knowledge}

\section{Paradigma Pemrograman}\label{paradigma-pemrograman}

Paradigma pemrograman adalah kerangka konseptual berpikir yang
mendefinisikan bagaimana seorang programmer memodelkan realitas,
mengorganisasi logika, dan mengoordinasikan perubahan. Setiap paradigma
menawarkan jawaban berbeda atas pertanyaan mendasar: apa itu komputasi,
di mana letak perubahan, dan siapa yang memegang kendali. Tidak ada satu
paradigma yang unggul secara absolut; masing-masing adalah instrumen
kognitif yang efektif dalam konteks tertentu.

\subsection{Paradigma Imperatif: Komputasi sebagai Urutan
Aksi}\label{paradigma-imperatif-komputasi-sebagai-urutan-aksi}

Paradigma imperatif memandang komputasi sebagai serangkaian perintah
eksplisit yang dieksekusi mesin langkah demi langkah. Dunia dipahami
sebagai keadaan (state) yang dapat diubah, dan pemrogram bertindak
sebagai pengendali langsung yang memberi instruksi tentang apa yang
harus dilakukan selanjutnya.

Kekuatan paradigma ini terletak pada:

\begin{itemize}
\tightlist
\item
  kejelasan kontrol,
\item
  kedekatan dengan cara kerja mesin,
\item
  dan efisiensi operasional.
\end{itemize}

Namun, beban kognitifnya tinggi. Tanggung jawab atas konsistensi state
sepenuhnya berada pada manusia. Kesalahan kecil dalam urutan aksi dapat
menghasilkan konsekuensi besar. Paradigma ini cocok untuk sistem tingkat
rendah dan situasi di mana kontrol presisi mutlak diperlukan.

\subsection{Paradigma Prosedural: Komputasi sebagai Struktur
Tertata}\label{paradigma-prosedural-komputasi-sebagai-struktur-tertata}

Paradigma prosedural adalah evolusi dari imperatif yang memperkenalkan
abstraksi struktural. Program tidak lagi dipahami sebagai aliran
instruksi mentah, tetapi sebagai kumpulan prosedur atau fungsi yang
terorganisasi.

Di sini, pemrogram mulai:

\begin{itemize}
\tightlist
\item
  memecah kompleksitas,
\item
  menamai pola perilaku,
\item
  dan membangun hierarki logika.
\end{itemize}

Paradigma ini menandai transisi penting: dari menyuruh mesin ke
mengorganisasi pikiran. Namun, meskipun struktur meningkat, state global
masih dominan, sehingga kompleksitas tetap tumbuh seiring skala sistem.

\subsection{Paradigma Berorientasi Objek: Komputasi sebagai Interaksi
Entitas}\label{paradigma-berorientasi-objek-komputasi-sebagai-interaksi-entitas}

Paradigma berorientasi objek (OOP) memodelkan dunia sebagai kumpulan
entitas otonom yang memiliki identitas, keadaan, dan perilaku. Komputasi
dipahami sebagai hasil dari interaksi antar-objek, bukan sekadar urutan
instruksi.

Keunggulan utama paradigma ini adalah:

\begin{itemize}
\tightlist
\item
  korespondensi intuitif dengan dunia nyata,
\item
  enkapsulasi kompleksitas,
\item
  dan modularitas sistem.
\end{itemize}

Namun, OOP membawa risiko konseptual: ketika relasi antar-objek tumbuh
tanpa disiplin, sistem dapat menjadi rapuh dan sulit dipahami. Paradigma
ini efektif ketika masalah memang bersifat entitas-sentris, tetapi
kurang optimal untuk transformasi data murni.

\subsection{Paradigma Fungsional: Komputasi sebagai Transformasi
Makna}\label{paradigma-fungsional-komputasi-sebagai-transformasi-makna}

Paradigma fungsional memandang komputasi sebagai transformasi nilai,
bukan manipulasi keadaan. Fungsi bersifat murni; input yang sama selalu
menghasilkan output yang sama. State, jika ada, diperlakukan sebagai
sesuatu yang harus dikendalikan secara ketat atau dieliminasi.

Paradigma ini memindahkan fokus dari bagaimana mesin bekerja ke apa
makna komputasi. Dampaknya adalah:

\begin{itemize}
\tightlist
\item
  kemudahan penalaran,
\item
  komposabilitas tinggi,
\item
  dan kesesuaian alami dengan paralelisme.
\end{itemize}

Paradigma fungsional menuntut kedewasaan berpikir. Ia kurang intuitif
bagi pemula, tetapi sangat kuat untuk sistem kompleks, konkuren, dan
berskala besar.

\subsection{Paradigma Deklaratif: Komputasi sebagai Pernyataan
Kebenaran}\label{paradigma-deklaratif-komputasi-sebagai-pernyataan-kebenaran}

Paradigma deklaratif membawa abstraksi lebih jauh dengan memisahkan niat
dari mekanisme. Pemrogram menyatakan apa yang diinginkan, bukan
bagaimana mencapainya. Mesin bertanggung jawab atas eksekusi.

SQL, logika predikat, dan constraint programming adalah contoh
manifestasi paradigma ini. Kekuatan utamanya:

\begin{itemize}
\tightlist
\item
  ekspresivitas tinggi,
\item
  fokus pada makna,
\item
  dan reduksi kompleksitas operasional.
\end{itemize}

Namun, kendali rendah atas proses dapat menjadi keterbatasan ketika
optimasi spesifik atau perilaku deterministik diperlukan.

\subsection{Sintesis Paradigmatik}\label{sintesis-paradigmatik}

Paradigma pemrograman bukanlah pilihan eksklusif, melainkan repertoar
konseptual. Seorang programmer yang matang tidak ``menganut'' satu
paradigma, tetapi memilih dan mengombinasikannya secara sadar sesuai
dengan sifat masalah.

Dengan demikian, paradigma pemrograman berfungsi sebagai:

\begin{itemize}
\tightlist
\item
  konsep induk bagi desain bahasa,
\item
  kerangka kognitif bagi pemecahan masalah,
\item
  dan alat pemberdayaan manusia dalam mengendalikan kompleksitas
  komputasional.
\end{itemize}

\bookmarksetup{startatroot}

\chapter{UAS-5 My Professional
Reviews}\label{uas-5-my-professional-reviews}

Untuk melAkukan review, seperti pada
\href{../My_Personal_Reviews/Doc.5.Mengevaluasi-Esai-Berdasarkan-Rubrik.pdf}{pendekatan
AI}, kita membutuhkan rubrik

\bookmarksetup{startatroot}

\chapter{Summary}\label{summary}

In summary, this book has no content whatsoever.

\bookmarksetup{startatroot}

\chapter*{References}\label{references}
\addcontentsline{toc}{chapter}{References}

\markboth{References}{References}

\phantomsection\label{refs}




\end{document}
