% Options for packages loaded elsewhere
% Options for packages loaded elsewhere
\PassOptionsToPackage{unicode}{hyperref}
\PassOptionsToPackage{hyphens}{url}
\PassOptionsToPackage{dvipsnames,svgnames,x11names}{xcolor}
%
\documentclass[
  letterpaper,
  DIV=11,
  numbers=noendperiod]{scrreprt}
\usepackage{xcolor}
\usepackage{amsmath,amssymb}
\setcounter{secnumdepth}{5}
\usepackage{iftex}
\ifPDFTeX
  \usepackage[T1]{fontenc}
  \usepackage[utf8]{inputenc}
  \usepackage{textcomp} % provide euro and other symbols
\else % if luatex or xetex
  \usepackage{unicode-math} % this also loads fontspec
  \defaultfontfeatures{Scale=MatchLowercase}
  \defaultfontfeatures[\rmfamily]{Ligatures=TeX,Scale=1}
\fi
\usepackage{lmodern}
\ifPDFTeX\else
  % xetex/luatex font selection
\fi
% Use upquote if available, for straight quotes in verbatim environments
\IfFileExists{upquote.sty}{\usepackage{upquote}}{}
\IfFileExists{microtype.sty}{% use microtype if available
  \usepackage[]{microtype}
  \UseMicrotypeSet[protrusion]{basicmath} % disable protrusion for tt fonts
}{}
\makeatletter
\@ifundefined{KOMAClassName}{% if non-KOMA class
  \IfFileExists{parskip.sty}{%
    \usepackage{parskip}
  }{% else
    \setlength{\parindent}{0pt}
    \setlength{\parskip}{6pt plus 2pt minus 1pt}}
}{% if KOMA class
  \KOMAoptions{parskip=half}}
\makeatother
% Make \paragraph and \subparagraph free-standing
\makeatletter
\ifx\paragraph\undefined\else
  \let\oldparagraph\paragraph
  \renewcommand{\paragraph}{
    \@ifstar
      \xxxParagraphStar
      \xxxParagraphNoStar
  }
  \newcommand{\xxxParagraphStar}[1]{\oldparagraph*{#1}\mbox{}}
  \newcommand{\xxxParagraphNoStar}[1]{\oldparagraph{#1}\mbox{}}
\fi
\ifx\subparagraph\undefined\else
  \let\oldsubparagraph\subparagraph
  \renewcommand{\subparagraph}{
    \@ifstar
      \xxxSubParagraphStar
      \xxxSubParagraphNoStar
  }
  \newcommand{\xxxSubParagraphStar}[1]{\oldsubparagraph*{#1}\mbox{}}
  \newcommand{\xxxSubParagraphNoStar}[1]{\oldsubparagraph{#1}\mbox{}}
\fi
\makeatother


\usepackage{longtable,booktabs,array}
\usepackage{calc} % for calculating minipage widths
% Correct order of tables after \paragraph or \subparagraph
\usepackage{etoolbox}
\makeatletter
\patchcmd\longtable{\par}{\if@noskipsec\mbox{}\fi\par}{}{}
\makeatother
% Allow footnotes in longtable head/foot
\IfFileExists{footnotehyper.sty}{\usepackage{footnotehyper}}{\usepackage{footnote}}
\makesavenoteenv{longtable}
\usepackage{graphicx}
\makeatletter
\newsavebox\pandoc@box
\newcommand*\pandocbounded[1]{% scales image to fit in text height/width
  \sbox\pandoc@box{#1}%
  \Gscale@div\@tempa{\textheight}{\dimexpr\ht\pandoc@box+\dp\pandoc@box\relax}%
  \Gscale@div\@tempb{\linewidth}{\wd\pandoc@box}%
  \ifdim\@tempb\p@<\@tempa\p@\let\@tempa\@tempb\fi% select the smaller of both
  \ifdim\@tempa\p@<\p@\scalebox{\@tempa}{\usebox\pandoc@box}%
  \else\usebox{\pandoc@box}%
  \fi%
}
% Set default figure placement to htbp
\def\fps@figure{htbp}
\makeatother





\setlength{\emergencystretch}{3em} % prevent overfull lines

\providecommand{\tightlist}{%
  \setlength{\itemsep}{0pt}\setlength{\parskip}{0pt}}



 


\KOMAoption{captions}{tableheading}
\makeatletter
\@ifpackageloaded{bookmark}{}{\usepackage{bookmark}}
\makeatother
\makeatletter
\@ifpackageloaded{caption}{}{\usepackage{caption}}
\AtBeginDocument{%
\ifdefined\contentsname
  \renewcommand*\contentsname{Table of contents}
\else
  \newcommand\contentsname{Table of contents}
\fi
\ifdefined\listfigurename
  \renewcommand*\listfigurename{List of Figures}
\else
  \newcommand\listfigurename{List of Figures}
\fi
\ifdefined\listtablename
  \renewcommand*\listtablename{List of Tables}
\else
  \newcommand\listtablename{List of Tables}
\fi
\ifdefined\figurename
  \renewcommand*\figurename{Figure}
\else
  \newcommand\figurename{Figure}
\fi
\ifdefined\tablename
  \renewcommand*\tablename{Table}
\else
  \newcommand\tablename{Table}
\fi
}
\@ifpackageloaded{float}{}{\usepackage{float}}
\floatstyle{ruled}
\@ifundefined{c@chapter}{\newfloat{codelisting}{h}{lop}}{\newfloat{codelisting}{h}{lop}[chapter]}
\floatname{codelisting}{Listing}
\newcommand*\listoflistings{\listof{codelisting}{List of Listings}}
\makeatother
\makeatletter
\makeatother
\makeatletter
\@ifpackageloaded{caption}{}{\usepackage{caption}}
\@ifpackageloaded{subcaption}{}{\usepackage{subcaption}}
\makeatother
\usepackage{bookmark}
\IfFileExists{xurl.sty}{\usepackage{xurl}}{} % add URL line breaks if available
\urlstyle{same}
\hypersetup{
  pdftitle={Arlow Emmanuel Hergara},
  pdfauthor={13523161 Arlow Emmanuel Hergara},
  colorlinks=true,
  linkcolor={blue},
  filecolor={Maroon},
  citecolor={Blue},
  urlcolor={Blue},
  pdfcreator={LaTeX via pandoc}}


\title{Arlow Emmanuel Hergara}
\usepackage{etoolbox}
\makeatletter
\providecommand{\subtitle}[1]{% add subtitle to \maketitle
  \apptocmd{\@title}{\par {\large #1 \par}}{}{}
}
\makeatother
\subtitle{Portfolio Asesmen II-2100 KIPP}
\author{13523161 Arlow Emmanuel Hergara}
\date{2025-10-31}
\begin{document}
\maketitle

\renewcommand*\contentsname{Table of contents}
{
\hypersetup{linkcolor=}
\setcounter{tocdepth}{2}
\tableofcontents
}

\bookmarksetup{startatroot}

\chapter*{Selamat Berjumpa}\label{selamat-berjumpa}
\addcontentsline{toc}{chapter}{Selamat Berjumpa}

\markboth{Selamat Berjumpa}{Selamat Berjumpa}

Nama saya adalah Arlow Emmanuel Hergara. Orang biasa yang sedang
mengeksplorasi hidup dan semua intrikasi-intraksinya sembari
mengembangkan diri menjadi versi yang terbaik.

Saya memiliki ketertarikan dalam pemrograman terutama dalam game
development. Karena itu saya masuk ke dalam Sekolah Teknik Elektro dan
Informatika sebagai mahasiswa S1 Teknik Informatika. Sekarang, saya
sudah berada dalam semester 5.

\bookmarksetup{startatroot}

\chapter{UTS-1 All About Me}\label{uts-1-all-about-me}

\section{Memahami Manusia}\label{memahami-manusia}

Tidak ada dari kita yang lahir mengetahui semuanya. Bahkan, kita masuk
ke dunia ini tanpa mengetahui apa-apa. Kita bertahan hidup hanya melalui
naluri dasar yang mendorong kita untuk melakukan aksi tertentu. Aksi
tersebut pun hanya dapat meminta bantuan dari manusia-manusia lain di
sekitar kita. Pada awalnya, semua manusia lahir tidak berdaya.

Namun, manusia adalah mahkluk yang terus belajar. Sejak awal kita lahir,
kita selalu mengobservasi dunia di sekitar kita. Kita mempelajari
bagaimana dunia itu bekerja. Kita juga coba berinteraksi dengan dunia
dan melihat bagaimana responnya. Dari yang kita amati kita dapatkan
pengetahuan dan dari yang kita lakukan kita dapatkan pengalaman.

Dalam mengeksplorasi dunia ini, kita menemukan bahwa dunia yang kita
alami berbeda antara satu orang dengan orang lain. Kita semua hidup
dalam dunia yang sama, tetapi dunia itu menunjukkan muka yang berbeda
pada masing-masing individu. Dengan demikian kita belajar hal-hal
berbeda, membangun pengetahun dan pengalaman masing-masing, dan menjadi
pribadi-pribadi yang unik. Tanpa menjalani keseluruhan hidup orang lain,
tidak mungkin kita memahami orang tersebut.

\section{Memahami Kesalahan}\label{memahami-kesalahan}

Dalam dunia ini, seringkali kita dengar kabar mengenai kesalahan yang
dilakukan oleh orang-orang. Terkadang, kita sendiri merasakan langsung
dampak dari kesalahan yang dilakukan oleh orang lain. Kesalahan ini
dapat menyakiti kita, bahkan hingga mendalam. Karena demikian, kita
dapat menanggap bahwa orang-orang tersebut merupakan orang yang buruk
dan memang berusaha untuk menyakiti kita.

Walaupun demikian, tidak ada orang di dunia ini yang secara aktif
berusaha menjadi pribadi yang buruk. Pribadi manusia tidak bisa
dilepaskan dari keadaan lingkungannya. Setiap orang berusaha melakukan
yang terbaik dengan semua informasi dan pengalaman yang dimilikinya.
Sayang saja, pilihan yang dibuat berdasarkan informasi dan pengalaman
yang diperoleh dalam suatu lingkungan belum tentu tepat dalam lingkungan
yang lain. Hal tersebut menggambarkan salah satu kejahatan utama dunia
ini. Manusia dipaksa membuat pilihan berdasarkan pengetahuan yang tidak
lengkap dan pilihan ini melekat menjadi pribadi barunya.

\section{Memahami Diri Sendiri}\label{memahami-diri-sendiri}

Pada dua subbab sebelumnya, saya sudah menggambarkan pandangan saya
terhadap dunia ini, khususnya pada sifat manusia. Namun, pandangan
tersebut hanya bagian kecil dari apa yang membuat diri saya. Agar dapat
mengenal saya dengan lebih lengkap, kita harus membahas detail-detail
yang lebih di permukaan dan mungkin kurang menarik mengenai saya. Untuk
menjelaskan hal tersebut, biarkan saya menggunakan nada yang lebih
kasual.

Halo, aku Arlow Emmanuel Hergara, orang yang biasa-biasa saja. Aku
adalah mahasiswa S1 Teknik Informatika di STEI ITB. Hobi-hobiku adalah
main game, makan, dan tidur. Selain dari hobi, aku juga memiliki minat
di bidang game development, software development, dan low level
programming. Aku juga memiliki kepribadian yang cukup unik (menurut
saya) dibandingkan dengan orang lain.

Aku adalah angkatan 2023 dari jurusan S1 Teknik Informatika di STEI ITB.
Saya masuk karena ketertarikan saya dengan programming. Dari sejak kecil
saya dipaparkan pada programming oleh ayahku dan hingga sekarang aku
masih tertarik pada hal-hal yang bisa saya buat. Namun jika aku perlu
jujur, aku merasa ketertarikan tersebut sudah tidak sekuat dulu semenjak
masuk ITB. Sepertinya tugas tidak berakhir yang aku kerjakan setiap hari
membuatku agak malas. Namun, aku tetap ingin semangat belajar mengenai
programming apapun \emph{challenge}-nya.

Hobi-hobiku (main game, makan, dan tidur) memang merupakan hobi yang
dimiliki banyak orang lain. Ketiga hal itu sering saya lakukan untuk
menghilangkan stres dari kesulitan sehari-hari (terkadang dilakuin
terlalu banyak sih). Jika harus spesifik, game yang aku banyak main
adalah game gacha, khususnya Genshin Impact dan game rhythm khususnya
osu! (ayo main bareng kalau ada yang main juga). Untuk makan, aku bisa
makan apa aja dan untuk tidur ya tidur.

Minat pertama yang aku dapatkan adalah game development. Bahkan game
development adalah hal yang membuat aku tertarik dengan programming. Aku
menjadi tertarik juga dalam software development setelah berusaha untuk
membuat program-program berhubungan dengan game development yang bukan
game. Minat aku terhadap low level programming muncul terakhir ketika
mencoba melakukan hal yang membutuhkannya. Dalam semua hal yang aku
minati, aku tertarik dengan bagaimana aku dapat membuat suatu sistem
bekerja seperti yang aku ingini.

Aku mengatakan bahwa kepribadianku unik tetapi aku tidak tahu seberapa
benar pernyataan itu. aku cenderung introvert yang tidak dapat
berinteraksi dengan siapa saja. Namun untuk orang yang aku kenal, aku
lumayan sering bercanda dan ngobrol. Aku juga seringkali melakukan
hal-hal yang tidak jelas yang orang lain dapat saja menganggap
mengesalkan. Sebelumnya aku sering takut melakukan hal-hal seperti itu
tapi seiringnya waktu aku belajar untuk tidak memikirkannya karena orang
pada umumnya tidak memerhatikan begitu banyak. Aku dulu orang yang suka
sedih, tetapi sekarang aku lagi berusaha menjadi seseorang yang lebih
ceria.

\bookmarksetup{startatroot}

\chapter{UTS-2 My Songs for You}\label{uts-2-my-songs-for-you}

\section{Musik yang Sedang Aku
Dengarkan}\label{musik-yang-sedang-aku-dengarkan}

\url{https://youtu.be/wWN7CaIQ2FU?si=lUU3BOX8Q6DYtetG}

\url{https://youtu.be/IZwVq37cOBk?si=cwOuogIIZ-2hfp7e}

\url{https://youtu.be/xPHjhcTK7gg?si=N-5LYuTnGL1Ehedq}

\section{Musik yang Aku Dengarkan Sejak
Lama}\label{musik-yang-aku-dengarkan-sejak-lama}

\url{https://youtu.be/pV1GoNi5mFs?si=CM5eZf3NFwQC1YW7}

\url{https://youtu.be/JprsKeAStcw?si=gu91EvJkcwhrBFEp}

\url{https://youtu.be/mpsDywA3wNI?si=ACt0uhPof_Js0E5I}

\bookmarksetup{startatroot}

\chapter{UTS-3 My Stories for You}\label{uts-3-my-stories-for-you}

\url{https://youtu.be/cn2hSXelQ0M?si=-xoBlNBDeCbjWpUF} Kisah ini
kebetulan saya temukan dalam bentuk MV sebuah musik. Namun saya tetap
menyukai kisah pengorbanan yang diceritakan dalamnya.

\bookmarksetup{startatroot}

\chapter{UTS-4 My SHAPE (Spiritual Gifts, Heart, Abilities, Personality,
Experiences)}\label{uts-4-my-shape-spiritual-gifts-heart-abilities-personality-experiences}

\section{SHAPE}\label{shape}

\subsection{S --- Spiritual Gifts (Karunia
Rohani)}\label{s-spiritual-gifts-karunia-rohani}

\textbf{Karunia utama}: Wisdom (hikmat) dan Teaching (mengajar) ---
kemampuan memahami struktur atau prinsip di balik suatu sistem, lalu
menerangkannya dengan logis agar orang lain mengerti dan bisa
menerapkannya.

\textbf{Kecenderungan alami}: suka merancang sistem atau konsep baru
(terutama lewat pemrograman dan desain logika), menemukan keteraturan
dari sesuatu yang rumit, serta menyalurkan wawasan itu lewat penjelasan
atau alat bantu (program, tulisan, atau diagram).

\textbf{Tanda-tanda karunia terlihat}:

\begin{itemize}
\tightlist
\item
  Orang lain sering meminta bantuanku untuk menjelaskan hal teknis atau
  memecahkan masalah logika.
\item
  Aku menikmati proses ``merancang cara berpikir'' --- bukan hanya hasil
  akhirnya.
\item
  Saat ideku berhasil membuat orang lain paham, aku merasa puas dan
  berarti.
\end{itemize}

\subsection{H --- Heart (Minat, Nilai, dan
Panggilan)}\label{h-heart-minat-nilai-dan-panggilan}

\textbf{Bidang yang dicintai}: teknologi, game, dan sistem berpikir yang
membuka kemungkinan baru. Nilai utama: eksplorasi, kebebasan berpikir,
dan membantu orang menemukan arah hidupnya.

\textbf{Panggilan pribadi}: menggunakan teknologi dan media interaktif
(seperti game) untuk memperluas cara orang belajar, berefleksi, dan
menemukan makna hidup.

\textbf{Masalah yang ingin diperbaiki}: banyak orang tersesat dalam
rutinitas tanpa arah, mereka butuh media yang menghidupkan kembali rasa
ingin tahu dan tujuan.

\textbf{Makna dalam karya}: ketika sebuah ide atau sistem yang saya buat
bisa memperluas kemungkinan, menyalakan rasa ingin tahu, dan memberi
jalan keluar bagi orang yang kehilangan arah.

\subsection{A --- Abilities (Kemampuan
Andal)}\label{a-abilities-kemampuan-andal}

\textbf{Kemampuan utama}:

\begin{itemize}
\tightlist
\item
  Pemrograman dalam berbagai bahasa dan paradigma (imperatif,
  fungsional, logika).
\item
  Desain sistem dan arsitektur perangkat lunak.
\item
  Debugging dan analisis kesalahan kompleks secara sistematis.
\item
  Koordinasi proyek dan pembagian tugas berbasis kemampuan tim.
\end{itemize}

\textbf{Gaya berpikir}: analitis dan terstruktur, mampu melihat
keterhubungan antarbagian dan mengoptimalkan sistem agar efisien dan
mudah dikembangkan.

\textbf{Kekuatan khas}: menyatukan konsep dari berbagai disiplin
(teknologi, logika, desain) menjadi satu sistem kerja yang elegan.

\textbf{Peran alami dalam tim}: desainer sistem dan penjaga kualitas,
orang yang memastikan proyek tetap konsisten dengan visi teknis dan
logika internalnya.

\subsection{P --- Personality (Kepribadian dan Gaya
Kerja)}\label{p-personality-kepribadian-dan-gaya-kerja}

\textbf{Gaya kerja}: fleksibel antara kerja mandiri dan kolaboratif,
mampu fokus mendalam saat dibutuhkan, tetapi juga siap berdiskusi untuk
mencari solusi yang paling rasional.

\textbf{Pola pengambilan keputusan}: berbasis logika dan prinsip, bukan
emosi sesaat. Sering memprioritaskan solusi ideal jangka panjang walau
memerlukan lebih banyak waktu dan energi.

\textbf{Kecenderungan khas}: perfeksionis terhadap sistem dan ide, ingin
setiap komponen memiliki alasan dan keterhubungan yang jelas.

\textbf{Sikap terhadap konflik}: tegas dan langsung menyampaikan
pendapat, terutama bila ada hal yang melanggar prinsip logika atau
efisiensi.

\textbf{Peran alami}: penalar strategis dan penjaga arah, memastikan
keputusan tidak sekadar cepat, tapi juga benar dalam jangka panjang.

\subsection{Experience (Pengalaman
Hidup)}\label{experience-pengalaman-hidup}

\textbf{Pengalaman pembentuk utama}: tantangan besar saat mencoba
membangun sistem operasi sendiri --- sebuah proyek yang menuntut
pemahaman mendalam tentang cara kerja komputer dari level paling dasar.

\textbf{Pelajaran dari kegagalan}: meski proyek itu belum berhasil,
pengalaman tersebut membentuk pola pikir sistematis dan menghargai
pentingnya fondasi yang kokoh dalam setiap rancangan. Kegagalan itu
menjadi pengingat bahwa visi besar membutuhkan kesabaran, disiplin, dan
kesediaan untuk memahami detail kecil sebelum membangun sesuatu yang
kompleks.

\textbf{Makna pribadi}: dari proses itu lahir tekad untuk terus memahami
bagaimana teknologi bekerja dari dalam, bukan sekadar menggunakannya.
Walau belum menemukan momen ``inilah panggilan saya,'' perjalanan
eksplorasi itu sendiri menjadi cara Anda mencari makna.

\section{Piagam Diri --- Arlow Emmanuel
Hergara}\label{piagam-diri-arlow-emmanuel-hergara}

\subsection{Misi Hidup (Life Mission)}\label{misi-hidup-life-mission}

Menjadi seseorang yang memajukan teknologi agar manusia dapat melakukan
hal-hal yang sebelumnya mustahil dilakukan --- menciptakan sistem yang
memperluas potensi, bukan membatasi.

\subsection{Nilai Inti (Core Values)}\label{nilai-inti-core-values}

\begin{enumerate}
\def\labelenumi{\arabic{enumi}.}
\tightlist
\item
  Rasionalitas dan kejujuran intelektual.
\item
  Empati --- kemajuan tidak boleh mengorbankan kesejahteraan orang lain.
\item
  Keberlanjutan --- setiap sistem harus dibangun agar tetap bermanfaat
  dan tidak merusak keseimbangan.
\end{enumerate}

\subsection{Prinsip Keputusan (Decision
Compass)}\label{prinsip-keputusan-decision-compass}

Saya akan menilai setiap tindakan dengan dua pertanyaan:

\begin{itemize}
\tightlist
\item
  Apakah ini masuk akal dan benar secara sistemik?
\item
  Apakah ini adil dan tidak memaksa orang untuk memberi lebih dari yang
  mereka miliki?
\end{itemize}

Jika suatu keputusan tidak lulus kedua pertanyaan ini, maka arah itu
bukan untuk saya.

\subsection{Peran Utama (Primary Role)}\label{peran-utama-primary-role}

Arsitek sistem dan penalar strategis --- seseorang yang merancang
fondasi logika dan teknologi agar orang lain dapat membangun di atasnya
dengan aman dan efisien.

\subsection{Gaya Pelayanan (Service
Style)}\label{gaya-pelayanan-service-style}

Mengajar dan membimbing melalui logika, menjelaskan hal sulit dengan
cara yang dapat dipahami, serta menciptakan alat yang membantu orang
lain memahami dunia mereka.

\subsection{Janji Hidup (Life Promise)}\label{janji-hidup-life-promise}

Saya berjanji untuk terus belajar, memahami dasar setiap hal, dan
menggunakan pemahaman itu untuk menciptakan teknologi yang memperluas
kebebasan manusia, bukan mempersempitnya.

\subsection{Batas Etis (Ethical
Boundaries)}\label{batas-etis-ethical-boundaries}

Saya tidak akan membuat sistem yang memaksa, mengeksploitasi, atau
menekan manusia demi efisiensi atau keuntungan. Teknologi harus melayani
manusia --- bukan sebaliknya.

\subsection{Narasi Diri (90 Detik)}\label{narasi-diri-90-detik}

Saya adalah seseorang yang mencintai logika dan sistem, dengan hasrat
untuk memahami bagaimana teknologi bekerja dan bagaimana ia dapat
memperluas kemampuan manusia. Saya menikmati proses merancang,
memprogram, dan membangun struktur yang tidak hanya efisien tetapi juga
adil bagi penggunanya. Saya percaya bahwa kemajuan sejati tidak boleh
memaksa siapa pun mengorbankan lebih dari yang mereka miliki. Pengalaman
saya mencoba membangun sistem operasi sendiri, meskipun belum berhasil,
mengajarkan pentingnya fondasi dan kesabaran dalam mewujudkan visi
besar. Melalui perjalanan ini, saya ingin terus berkontribusi dalam
pengembangan teknologi yang membantu manusia melakukan hal-hal yang
sebelumnya tidak mungkin, sambil tetap menjaga nilai-nilai kemanusiaan
di dalamnya.

\bookmarksetup{startatroot}

\chapter{UTS-5 My Personal Reviews}\label{uts-5-my-personal-reviews}

\section{Identifikasi}\label{identifikasi}

\begin{enumerate}
\def\labelenumi{\arabic{enumi}.}
\tightlist
\item
  \textbf{Nama Mahasiswa TUGAS:} Arlow Emmanuel Hergara (berdasarkan
  konten UTS-1 dan UTS-4)
\item
  \textbf{Nama Penilai:} Self-Assessment (Gemini)
\end{enumerate}

\begin{center}\rule{0.5\linewidth}{0.5pt}\end{center}

\section{Tinjauan Umum}\label{tinjauan-umum}

Portofolio UTS ini menunjukkan kualitas yang sangat bervariasi. Terdapat
kekuatan luar biasa dalam refleksi diri dan pemikiran terstruktur
(ditunjukkan pada UTS-1 dan UTS-4), yang menunjukkan kedalaman pemahaman
dan orisinalitas.

Namun, TUGAS UTS-2, UTS-3, dan UTS-5 tampaknya salah memahami inti dari
tugas tersebut. Alih-alih membuat atau menyampaikan ``pesan personal'',
tugas-tugas tersebut sebagian besar hanya berisi tautan ke karya orang
lain (UTS-2 \& UTS-3) atau mendelegasikan tugas analisis (UTS-5) ke AI
tanpa ada kontribusi analitis pribadi.

\begin{center}\rule{0.5\linewidth}{0.5pt}\end{center}

\section{Tinjauan Spesifik \& Skor
Persentase}\label{tinjauan-spesifik-skor-persentase}

Berikut adalah penilaian rinci untuk setiap TUGAS:

\subsection{UTS-1: All About Me}\label{uts-1-all-about-me-1}

\begin{itemize}
\tightlist
\item
  \textbf{Tinjauan:} Konten ini luar biasa. Dimulai dengan perenungan
  filosofis tentang ``Memahami Manusia'' dan ``Memahami Kesalahan''
  sebelum beralih ke perkenalan diri yang lebih kasual. Pendekatan ini
  sangat orisinal dan memberikan wawasan mendalam.
\item
  \textbf{Penilaian (Rubrik UTS-1):}

  \begin{itemize}
  \tightlist
  \item
    Orisinalitas (5/5): Sudut pandang sangat unik.
  \item
    Keterlibatan (4/5): Umumnya menarik.
  \item
    Humor (2/5): Kurang. Nada tulisan cenderung serius.
  \item
    Wawasan (Insight) (5/5): Memberi pemahaman mendalam.
  \end{itemize}
\item
  \textbf{Skor Persentase: 16 / 20 (80\%)}
\end{itemize}

\begin{center}\rule{0.5\linewidth}{0.5pt}\end{center}

\subsection{UTS-2: My Song for You}\label{uts-2-my-song-for-you}

\begin{itemize}
\tightlist
\item
  \textbf{Tinjauan:} Tugas ini meminta ``pesan berbentuk puisi, lago,
  dan/atau viodeo clip''{[}cite: 17{]}. Konten yang dikumpulkan hanya
  berupa enam tautan video YouTube tanpa konteks atau pesan pribadi.
\item
  \textbf{Penilaian (Rubrik UTS-2):}

  \begin{itemize}
  \tightlist
  \item
    Orisinalitas (1/5): Klise. Hanya daftar tautan.
  \item
    Keterlibatan (1/5): Tidak memikat.
  \item
    Humor (1/5): Tidak ada.
  \item
    Inspirasi (1/5): Tidak menginspirasi.
  \end{itemize}
\item
  \textbf{Skor Persentase: 4 / 20 (20\%)}
\end{itemize}

\begin{center}\rule{0.5\linewidth}{0.5pt}\end{center}

\subsection{UTS-3: My Stories for You}\label{uts-3-my-stories-for-you-1}

\begin{itemize}
\tightlist
\item
  \textbf{Tinjauan:} Tugas ini meminta ``kisah inspiratif dan menarik
  yang Anda ingin bagikan''{[}cite: 18{]}. Konten yang dikumpulkan
  adalah satu video musik dengan komentar singkat. Penulis tidak
  membagikan \emph{kisah mereka sendiri}.
\item
  \textbf{Penilaian (Rubrik UTS-3):}

  \begin{itemize}
  \tightlist
  \item
    Orisinalitas (1/5): Tidak ada pengembangan baru.
  \item
    Keterlibatan (1/5): Tidak menarik (dari sisi penulis).
  \item
    Pengembangan Narasi (1/5): Terputus dari cerita (personal).
  \item
    Inspirasi (2/5): Berusaha menginspirasi, namun dangkal.
  \end{itemize}
\item
  \textbf{Skor Persentase: 5 / 20 (25\%)}
\end{itemize}

\begin{center}\rule{0.5\linewidth}{0.5pt}\end{center}

\subsection{UTS-4: My SHAPE}\label{uts-4-my-shape}

\begin{itemize}
\tightlist
\item
  \textbf{Tinjauan:} Tugas ini meminta ``laporan siapa Anda berdasar
  hasil sebuah lembar kerja''{[}cite: 20{]}. Konten yang disajikan
  sangat baik, rinci, dan terstruktur. Penulis mengembangkannya menjadi
  ``Piagam Diri'' dan ``Narasi Diri''.
\item
  \textbf{Penilaian (Rubrik UTS-4):} \emph{(Rubrik di PDF tampaknya
  salah salin, penilaian ini menginterpretasikan kriteria dalam konteks
  laporan SHAPE).}

  \begin{itemize}
  \tightlist
  \item
    Orisinalitas (5/5): Pengembangan laporan sangat unik.
  \item
    Keterlibatan (5/5): Sangat memikat dan terstruktur.
  \item
    Pengembangan Narasi (5/5): Logis, dari analisis SHAPE ke Piagam
    Diri.
  \item
    Inspirasi (5/5): Sangat menginspirasi (mis. Misi Hidup, Batas Etis).
  \end{itemize}
\item
  \textbf{Skor Persentase: 20 / 20 (100\%)}
\end{itemize}

\begin{center}\rule{0.5\linewidth}{0.5pt}\end{center}

\subsection{UTS-5: My Personal
Reviews}\label{uts-5-my-personal-reviews-1}

\begin{itemize}
\tightlist
\item
  \textbf{Tinjauan:} Tugas ini adalah ``telaahan pesan personal
  berdasarkan rubrik''{[}cite: 21{]}. Konten yang dikumpulkan
  \emph{bukanlah} telaahan yang ditulis oleh penulis. Penulis
  mendokumentasikan proses menggunakan ChatGPT untuk menilai portofolio
  \emph{orang lain}.
\item
  \textbf{Penilaian (Rubrik UTS-5):}

  \begin{itemize}
  \tightlist
  \item
    Pemahaman Konsep (1/5): Tidak paham. Tugas didelegasikan.
  \item
    Analisis Kritis (1/5): Tidak kritis. Tidak ada analisis pribadi.
  \item
    Argumentasi (Logos) (1/5): Tidak logis.
  \item
    Etos \& Empati (1/5): Tidak tampak.
  \item
    Rekomendasi Perbaikan (1/5): Tidak ada (dari penulis).
  \end{itemize}
\item
  \textbf{Skor Persentase: 5 / 25 (20\%)}
\end{itemize}

\begin{center}\rule{0.5\linewidth}{0.5pt}\end{center}

\section{SKOR (Perhitungan Kontribusi
CPMK)}\label{skor-perhitungan-kontribusi-cpmk}

Berdasarkan \texttt{Tabel\ 1}, bobot untuk setiap tugas UTS adalah
sebagai berikut: * \textbf{UTS-1:} Bobot 6 (untuk CPMK-2) *
\textbf{UTS-2:} Bobot 7 (untuk CPMK-2) * \textbf{UTS-3:} Bobot 7 (untuk
CPMK-2) * \textbf{UTS-4:} Bobot 6 (untuk CPMK-2) * \textbf{UTS-5:} Bobot
10 (untuk CPMK-1)

Perhitungan kontribusi:
\texttt{(Skor\ Persentase\ /\ 100)\ *\ Bobot\ Tugas}

\begin{itemize}
\tightlist
\item
  \textbf{UTS-1 (CPMK-2):} (80\% / 100) * 6 = \textbf{4.8}
\item
  \textbf{UTS-2 (CPMK-2):} (20\% / 100) * 7 = \textbf{1.4}
\item
  \textbf{UTS-3 (CPMK-2):} (25\% / 100) * 7 = \textbf{1.75}
\item
  \textbf{UTS-4 (CPMK-2):} (100\% / 100) * 6 = \textbf{6.0}
\item
  \textbf{UTS-5 (CPMK-1):} (20\% / 100) * 10 = \textbf{2.0}
\end{itemize}

Tabel ini diisi sesuai format laporan{[}cite: 65{]}:

\begin{longtable}[]{@{}lcccc@{}}
\toprule\noalign{}
UTS & CPMK-1 & CPMK-2 & CPMK-3 & CPMK-4 \\
\midrule\noalign{}
\endhead
\bottomrule\noalign{}
\endlastfoot
UTS-1 & - & 4.8 & - & - \\
UTS-2 & - & 1.4 & - & - \\
UTS-3 & - & 1.75 & - & - \\
UTS-4 & - & 6.0 & - & - \\
UTS-5 & 2.0 & - & - & - \\
\textbf{Total Kontribusi} & \textbf{2.0} & \textbf{13.95} & \textbf{0} &
\textbf{0} \\
\emph{(Total Bobot UTS)} & \emph{(10)} & \emph{(26)} & \emph{(0)} &
\emph{(0)} \\
\end{longtable}

\bookmarksetup{startatroot}

\chapter{UAS-1 My Concepts}\label{uas-1-my-concepts}

Konsep yang ingin saya bawa adalah Learning Model Artificial
Intelligence (AI). Konsep ini memandang kecerdasan buatan sebagai sistem
yang tidak hanya menjalankan instruksi tetap, tetapi mampu belajar dari
pengalaman dan memperbaiki perilakunya seiring waktu.

Berbeda dengan sistem AI yang sepenuhnya berbasis aturan, Learning Model
AI memiliki kemampuan untuk mengevaluasi hasil tindakannya sendiri.
Ketika sebuah tindakan menghasilkan hasil yang kurang optimal, sistem
dapat menyesuaikan model internalnya agar di masa depan dapat membuat
keputusan yang lebih baik. Dengan cara ini, kecerdasan tidak berhenti
pada desain awal, melainkan berkembang melalui proses pembelajaran
berkelanjutan.

Learning Model AI bekerja melalui siklus sederhana namun berulang.
Sistem menerima informasi dari lingkungan, mengambil tindakan
berdasarkan pengetahuan yang dimilikinya, lalu menerima umpan balik atas
tindakan tersebut. Umpan balik ini menjadi dasar bagi sistem untuk
memperbarui cara kerjanya. Proses inilah yang memungkinkan AI
beradaptasi terhadap perubahan dan ketidakpastian.

Konsep ini juga mengubah hubungan antara manusia dan mesin. Manusia
tidak lagi harus menentukan setiap langkah yang harus diambil oleh
sistem, melainkan menetapkan tujuan, batasan, dan kriteria keberhasilan.
Mesin diberi ruang untuk mengeksplorasi solusi di dalam kerangka
tersebut dan belajar dari hasil yang diperoleh.

\bookmarksetup{startatroot}

\chapter{UAS-3 My Opinions}\label{uas-3-my-opinions}

Model Artificial Intelligence yang populer saat ini, khususnya Large
Language Models (LLM), sangat bergantung pada pendekatan pembelajaran
berbasis big data. Kapabilitas utama model-model ini diperoleh melalui
proses pelatihan awal yang masif, menggunakan kumpulan data dalam skala
sangat besar, dengan tujuan menangkap pola statistik dari bahasa dan
pengetahuan manusia. Pendekatan ini terbukti efektif untuk berbagai
tugas praktis, namun menimbulkan keterbatasan mendasar ketika dikaitkan
dengan tujuan jangka panjang Artificial General Intelligence (AGI).

Ketergantungan yang tinggi pada big data menunjukkan bahwa pembelajaran
LLM bersifat front-loaded, yaitu sebagian besar kecerdasan model
ditentukan pada fase pelatihan awal. Setelah model dirilis dan
digunakan, kemampuannya pada dasarnya bersifat statis. Model tidak
benar-benar belajar dari interaksi sehari-hari, tidak membangun
pemahaman baru secara berkelanjutan, dan tidak melakukan eksperimen
mandiri terhadap lingkungannya. Dengan demikian, penggunaan model lebih
menyerupai proses inferensi dari pengetahuan masa lalu, bukan
pembelajaran aktif yang sedang berlangsung.

Kondisi ini berbeda secara fundamental dengan cara kecerdasan umum
terbentuk pada manusia. Manusia tidak bergantung pada kumpulan data
raksasa yang dikurasi di awal kehidupan, melainkan belajar melalui
eksperimen, kegagalan, refleksi, dan adaptasi berkelanjutan.
Pembelajaran terjadi sepanjang penggunaan kemampuan itu sendiri, bukan
berhenti setelah fase ``pelatihan''.

Oleh karena itu, dapat dikemukakan bahwa model AI yang hanya
mengandalkan pembelajaran statistik dari big data bukanlah pendekatan
yang optimal untuk mencapai AGI. AGI menuntut sistem yang memiliki
kapasitas untuk membentuk pengetahuan baru secara otonom, menguji
hipotesisnya sendiri, dan memperbarui representasi internalnya
berdasarkan pengalaman langsung. Model semacam ini harus mampu belajar
in situ, seiring dengan penggunaan, bukan hanya sebelum digunakan.

Dengan kata lain, arah menuju AGI memerlukan pergeseran paradigma: dari
model yang terutama mengonsumsi data historis, menuju model yang mampu
bereksperimen secara mandiri dan belajar secara terus-menerus. Tanpa
kemampuan ini, AI akan tetap berada pada level kecerdasan
sempit---sangat kompeten dalam domain tertentu, tetapi tidak benar-benar
adaptif atau general.

\bookmarksetup{startatroot}

\chapter{UAS-3 My Innovations}\label{uas-3-my-innovations}

Inovasi yang saya kembangkan adalah \textbf{Continuous Learning Model},
yaitu sebuah pendekatan pemodelan AI yang dirancang untuk belajar secara
berkelanjutan selama digunakan, bukan hanya pada tahap awal
pengembangan. Dalam pendekatan ini, proses pembelajaran tidak dipisahkan
ke dalam fase training dan deployment yang kaku, melainkan berlangsung
secara terus-menerus seiring interaksi model dengan lingkungannya.
Dengan demikian, kemampuan model berkembang secara bertahap berdasarkan
pengalaman nyata, bukan semata-mata dari kumpulan data historis yang
besar.

Continuous Learning Model ini dibangun di atas beberapa komponen inti
yang saling melengkapi, yaitu sistem memori berlapis, pembelajaran
berbasis penguatan, mekanisme imajinasi, dan penjadwalan tugas.

\section{Memori N-Tier}\label{memori-n-tier}

Memori N-Tier merupakan mekanisme yang membagi memori model ke dalam
beberapa tingkatan (tier), di mana setiap tier merepresentasikan tingkat
kejelasan (clarity) dan stabilitas pengetahuan yang berbeda. Konsep ini
terinspirasi dari cara manusia membedakan antara pemikiran aktif,
pengalaman jangka pendek, dan intuisi dasar.

Tier dengan tingkat tertinggi berfungsi sebagai working memory, yaitu
memori dengan clarity paling tinggi yang digunakan secara langsung dalam
proses penalaran dan pengambilan keputusan. Memori pada tingkat ini
bersifat sangat dinamis dan mudah berubah, karena ia menangani konteks
dan informasi sementara.

Sebaliknya, tier memori yang lebih rendah menyimpan pengetahuan yang
lebih abstrak dan umum, hingga pada tingkat terendah berperan sebagai
instinct, yaitu pola perilaku dasar dengan clarity rendah namun stabil.
Tier yang lebih rendah lebih sulit untuk diubah, sehingga berfungsi
sebagai fondasi perilaku model.

Pembagian ini memungkinkan proses pembelajaran berlangsung secara
bertahap: pengalaman baru terlebih dahulu memengaruhi memori tingkat
atas, dan hanya setelah terbukti konsisten dan berguna, ia dapat
dikonsolidasikan ke tier yang lebih rendah dan lebih stabil.

\section{Reinforcement Learning}\label{reinforcement-learning}

Untuk memungkinkan pembelajaran selama penggunaan, model dirancang
menggunakan pendekatan reinforcement learning, di mana perilaku model
dipengaruhi oleh hasil interaksi dengan pengguna. Setiap interaksi
menghasilkan sinyal umpan balik yang berfungsi sebagai reward atau
penalty.

Interaksi yang dinilai positif, seperti jawaban yang diterima atau
digunakan oleh pengguna, akan memperkuat kecenderungan model untuk
mengulangi pola perilaku tersebut. Sebaliknya, interaksi negatif,
seperti koreksi atau penolakan, akan melemahkan kecenderungan tersebut.
Pembelajaran ini tidak dilakukan secara agresif, melainkan secara
gradual, sehingga menjaga stabilitas sistem.

Pendekatan ini memungkinkan model untuk menyesuaikan perilakunya
berdasarkan pengalaman nyata, tanpa memerlukan dataset berlabel secara
eksplisit.

\section{Imagination}\label{imagination}

Komponen Imagination merepresentasikan kemampuan model untuk melakukan
pemrosesan internal di luar interaksi langsung dengan pengguna. Proses
ini terjadi ketika model tidak sedang mengerjakan tugas eksplisit dan
berfungsi sebagai mekanisme refleksi serta pengembangan pengetahuan.

Dalam mode imajinasi, model mengombinasikan informasi yang telah
dimilikinya untuk: - membentuk hipotesis baru, - mengevaluasi alternatif
respons, - dan mensimulasikan kemungkinan hasil dari strategi tertentu.

Melalui proses ini, model dapat melakukan eksperimen internal tanpa
risiko langsung terhadap pengguna, sehingga memungkinkan pengembangan
pengetahuan dan perilaku secara aman dan terkontrol.

\section{Task Scheduling}\label{task-scheduling}

Berbeda dari model konvensional yang bekerja dalam pola
request--response instan, Continuous Learning Model menggunakan
mekanisme task scheduling. Setiap permintaan pengguna dimasukkan ke
dalam antrian tugas yang dapat diproses secara bertahap seiring waktu.

Pendekatan ini memberikan model ruang untuk melakukan: - perencanaan, -
penalaran bertahap, - evaluasi ulang terhadap hasil sementara.

Dengan adanya task scheduling, model dapat mengiterasi pekerjaannya
sendiri, memperbaiki kesalahan, dan meningkatkan kualitas hasil secara
progresif. Pendekatan ini meniru cara manusia menyelesaikan masalah
kompleks, di mana pemikiran dan perbaikan tidak selalu terjadi secara
instan.

\bookmarksetup{startatroot}

\chapter{UAS-4 My Knowledge}\label{uas-4-my-knowledge}

Pendekatan pembuatan learning model untuk artificial intelligence yang
paling umum digunakan sekarang adalah pendekatan supervised dan
unsupervised.

\section{Supervised Learning}\label{supervised-learning}

Pada supervised learning, model belajar dari pasangan input--output yang
telah diberi label. Tujuan utamanya adalah mempelajari fungsi pemetaan
dari data contoh.

Model-model yang menggunakan Supervised Learning

\subsection{Linear Models}\label{linear-models}

Digunakan untuk masalah prediksi sederhana.

\begin{itemize}
\tightlist
\item
  Linear Regression
\item
  Logistic Regression
\end{itemize}

\subsection{Tree-Based Models}\label{tree-based-models}

Belajar melalui pemisahan keputusan berbasis fitur.

\begin{itemize}
\tightlist
\item
  Decision Tree
\item
  Random Forest
\item
  Gradient Boosting (XGBoost, LightGBM)
\end{itemize}

\subsection{Support Vector Machine
(SVM)}\label{support-vector-machine-svm}

Mencari batas keputusan optimal antar kelas dengan margin maksimum.

\begin{itemize}
\tightlist
\item
  Neural Networks (Supervised): Digunakan ketika data dan hubungan antar
  fitur kompleks.
\item
  Multilayer Perceptron (MLP)
\item
  Convolutional Neural Network (CNN) untuk visi komputer
\item
  Recurrent Neural Network (RNN), LSTM untuk data sekuensial
\end{itemize}

\section{Unsupervised Learning}\label{unsupervised-learning}

Unsupervised learning bekerja tanpa label eksplisit. Model bertujuan
menemukan struktur laten atau pola tersembunyi dalam data.

Model-model yang menggunakan Unsupervised Learning

\subsection{Clustering Models}\label{clustering-models}

Mengelompokkan data berdasarkan kemiripan.

\begin{itemize}
\tightlist
\item
  K-Means
\item
  Hierarchical Clustering
\item
  DBSCAN
\end{itemize}

\subsection{Dimensionality Reduction
Models}\label{dimensionality-reduction-models}

Menyederhanakan representasi data.

\begin{itemize}
\tightlist
\item
  Principal Component Analysis (PCA)
\item
  t-SNE
\item
  UMAP
\end{itemize}

\subsection{Probabilistic Models}\label{probabilistic-models}

Memodelkan distribusi data.

\begin{itemize}
\tightlist
\item
  Gaussian Mixture Model (GMM)
\item
  Hidden Markov Model (tanpa label eksplisit)
\end{itemize}

\subsection{Representation Learning
Models}\label{representation-learning-models}

Belajar representasi internal data.

\begin{itemize}
\tightlist
\item
  Autoencoder
\item
  Variational Autoencoder (VAE)
\end{itemize}

\bookmarksetup{startatroot}

\chapter{UAS-5 My Professional
Reviews}\label{uas-5-my-professional-reviews}

\section{Review: My Concepts}\label{review-my-concepts}

Judul Konsep: Learning Model Artificial Intelligence

Analisis: Esai ini mengajukan konsep bahwa AI tidak boleh sekadar
berbasis aturan atau statis, melainkan harus berupa sistem yang belajar
dari umpan balik untuk memperbaiki model internalnya secara
berkelanjutan.

\begin{longtable}[]{@{}
  >{\raggedright\arraybackslash}p{(\linewidth - 6\tabcolsep) * \real{0.0179}}
  >{\raggedright\arraybackslash}p{(\linewidth - 6\tabcolsep) * \real{0.0072}}
  >{\raggedright\arraybackslash}p{(\linewidth - 6\tabcolsep) * \real{0.5224}}
  >{\raggedright\arraybackslash}p{(\linewidth - 6\tabcolsep) * \real{0.4526}}@{}}
\toprule\noalign{}
\begin{minipage}[b]{\linewidth}\raggedright
Kriteria
\end{minipage} & \begin{minipage}[b]{\linewidth}\raggedright
Skor
\end{minipage} & \begin{minipage}[b]{\linewidth}\raggedright
Predikat
\end{minipage} & \begin{minipage}[b]{\linewidth}\raggedright
Analisis Penilaian
\end{minipage} \\
\midrule\noalign{}
\endhead
\bottomrule\noalign{}
\endlastfoot
Clarity & 5 & Excellent & Konsep disajikan dengan sangat jelas. Penulis
mendefinisikan perbedaan antara AI berbasis aturan (rule-based) dan
Learning Model dengan bahasa yang tegas dan tidak ambigu. Penjelasan
mengenai siklus ``informasi -\textgreater{} tindakan -\textgreater{}
umpan balik'' mudah dipahami. \\
Logic & 5 & Excellent & Logika yang dibangun sangat koheren. Argumen
mengalir dari definisi masalah (hasil kurang optimal), solusi
(penyesuaian model internal), hingga implikasi (hubungan
manusia-mesin). \\
Validity & 5 & Excellent & Konsep ini valid dan didukung oleh prinsip
dasar Machine Learning modern (khususnya Reinforcement Learning). Premis
bahwa mesin perlu ``ruang untuk mengeksplorasi'' agar bisa beradaptasi
adalah penalaran yang kuat (sound reasoning). \\
Usefulness & 5 & Excellent & Konsep ini memiliki kegunaan praktis yang
signifikan karena menawarkan solusi untuk masalah ketidakpastian dan
perubahan lingkungan yang tidak bisa ditangani oleh sistem statis. \\
\end{longtable}

\section{Review: My Opinions}\label{review-my-opinions}

Topik: Keterbatasan LLM dan Kebutuhan Menuju AGI (Artificial General
Intelligence)

Analisis: Penulis beropini bahwa model LLM saat ini bersifat
front-loaded (cerdas di awal pelatihan) dan statis, sehingga tidak
optimal untuk mencapai AGI yang menuntut pembelajaran mandiri (in situ).

\begin{longtable}[]{@{}
  >{\raggedright\arraybackslash}p{(\linewidth - 6\tabcolsep) * \real{0.0179}}
  >{\raggedright\arraybackslash}p{(\linewidth - 6\tabcolsep) * \real{0.0072}}
  >{\raggedright\arraybackslash}p{(\linewidth - 6\tabcolsep) * \real{0.5224}}
  >{\raggedright\arraybackslash}p{(\linewidth - 6\tabcolsep) * \real{0.4526}}@{}}
\toprule\noalign{}
\begin{minipage}[b]{\linewidth}\raggedright
Kriteria
\end{minipage} & \begin{minipage}[b]{\linewidth}\raggedright
Skor
\end{minipage} & \begin{minipage}[b]{\linewidth}\raggedright
Predikat
\end{minipage} & \begin{minipage}[b]{\linewidth}\raggedright
Analisis Penilaian
\end{minipage} \\
\midrule\noalign{}
\endhead
\bottomrule\noalign{}
\endlastfoot
Compelling & 5 & Excellent & Esai ini memikat karena berani mengkritik
tren populer saat ini (LLM \& Big Data) sebagai pendekatan yang memiliki
``keterbatasan mendasar'' untuk tujuan jangka panjang AGI. \\
Informative & 5 & Excellent & Memberikan wawasan mendalam (deep
insights) mengenai perbedaan fundamental cara belajar manusia
(eksperimen, kegagalan) dibandingkan dengan cara kerja LLM (inferensi
statis). \\
Persuasive & 5 & Excellent & Argumen sangat persuasif dengan menyatakan
bahwa AGI memerlukan ``pergeseran paradigma'' dari konsumsi data
historis menuju eksperimen mandiri. Penalaran ini didukung oleh
perbandingan logis yang kuat. \\
Engaging & 4 & Good & Esai melibatkan pembaca secara intelektual untuk
memikirkan ulang definisi ``belajar''. Namun, gaya bahasanya sangat
teknis dan akademis, mungkin sedikit kurang emosional dibanding esai
yang bersifat sosial. \\
\end{longtable}

\section{Review: My Innovations}\label{review-my-innovations}

Inovasi: Continuous Learning Model (Memori N-Tier, Imagination, Task
Scheduling)

Analisis: Inovasi yang diajukan adalah arsitektur AI yang belajar
terus-menerus tanpa pemisahan fase training dan deployment, didukung
oleh fitur memori bertingkat dan kemampuan imajinasi.

\begin{longtable}[]{@{}
  >{\raggedright\arraybackslash}p{(\linewidth - 4\tabcolsep) * \real{0.0327}}
  >{\raggedright\arraybackslash}p{(\linewidth - 4\tabcolsep) * \real{0.0131}}
  >{\raggedright\arraybackslash}p{(\linewidth - 4\tabcolsep) * \real{0.9542}}@{}}
\toprule\noalign{}
\begin{minipage}[b]{\linewidth}\raggedright
Kriteria
\end{minipage} & \begin{minipage}[b]{\linewidth}\raggedright
Skor
\end{minipage} & \begin{minipage}[b]{\linewidth}\raggedright
Analisis Penilaian (Berdasarkan Kriteria Dokumen Penilaian)
\end{minipage} \\
\midrule\noalign{}
\endhead
\bottomrule\noalign{}
\endlastfoot
Kurasi & 4 & Kurasi materi cukup baik dan mencakup algoritma-algoritma
utama yang menjadi standar industri (seperti XGBoost, CNN, LSTM,
K-Means). \\
Kejelasan & 5 & Definisi setiap kategori sangat ringkas dan jelas.
Contoh: Supervised dijelaskan sebagai ``belajar dari pasangan
input--output berlabel'' dan Unsupervised sebagai ``menemukan struktur
laten''. \\
Akurasi & 5 & Informasi teknis sangat akurat. Pengelompokan algoritma
sudah tepat (misal: PCA masuk ke Dimensionality Reduction, Random Forest
masuk ke Tree-Based). \\
Daya Guna & 4 & Berguna sebagai referensi cepat (cheat sheet) atau peta
konsep bagi pembelajar AI. Namun, karena sifatnya yang berupa daftar
(list), kedalaman analisis per poinnya terbatas. \\
\end{longtable}

\section{Review: My Knowledge}\label{review-my-knowledge}

Topik: Supervised vs Unsupervised Learning Models

Analisis: Dokumen ini merupakan kurasi pengetahuan teknis yang memetakan
algoritma pembelajaran mesin ke dalam kategori Supervised (Linear, Tree,
SVM, Neural Network) dan Unsupervised (Clustering, Dim-Reduction,
Probabilistic).

\begin{longtable}[]{@{}
  >{\raggedright\arraybackslash}p{(\linewidth - 4\tabcolsep) * \real{0.0327}}
  >{\raggedright\arraybackslash}p{(\linewidth - 4\tabcolsep) * \real{0.0131}}
  >{\raggedright\arraybackslash}p{(\linewidth - 4\tabcolsep) * \real{0.9542}}@{}}
\toprule\noalign{}
\begin{minipage}[b]{\linewidth}\raggedright
Kriteria
\end{minipage} & \begin{minipage}[b]{\linewidth}\raggedright
Skor
\end{minipage} & \begin{minipage}[b]{\linewidth}\raggedright
Analisis Penilaian (Berdasarkan Kriteria Dokumen Penilaian)
\end{minipage} \\
\midrule\noalign{}
\endhead
\bottomrule\noalign{}
\endlastfoot
Kurasi & 4 & Kurasi materi cukup baik dan mencakup algoritma-algoritma
utama yang menjadi standar industri (seperti XGBoost, CNN, LSTM,
K-Means). \\
Kejelasan & 5 & Definisi setiap kategori sangat ringkas dan jelas.
Contoh: Supervised dijelaskan sebagai ``belajar dari pasangan
input--output berlabel'' dan Unsupervised sebagai ``menemukan struktur
laten''. \\
Akurasi & 5 & Informasi teknis sangat akurat. Pengelompokan algoritma
sudah tepat (misal: PCA masuk ke Dimensionality Reduction, Random Forest
masuk ke Tree-Based). \\
Daya Guna & 4 & Berguna sebagai referensi cepat (cheat sheet) atau peta
konsep bagi pembelajar AI. Namun, karena sifatnya yang berupa daftar
(list), kedalaman analisis per poinnya terbatas. \\
\end{longtable}

\bookmarksetup{startatroot}

\chapter{Summary}\label{summary}

In summary, this book has no content whatsoever.

\bookmarksetup{startatroot}

\chapter*{References}\label{references}
\addcontentsline{toc}{chapter}{References}

\markboth{References}{References}

\phantomsection\label{refs}




\end{document}
